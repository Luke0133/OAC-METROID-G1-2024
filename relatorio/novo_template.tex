% 9pt -> fonte pequena
% twocolumn -> documento em duas colunas
% extarticle -> extensão do article, classe propria para artigos/relatório
\documentclass[9pt, twocolumn]{extarticle}
% Thiago T. P, 2022/2
% template original: https://www.overleaf.com/read/yfjhddwvpcgh
%%%%%%%%%%%%%%%%%%%%%%%%%%%%%%%%%%%%%%%%%%%%%%%%%%%%%%%%%%%%%%%%%
%% LEMBRE-SE DE INCLUIR OS ARQUIVOS preâmbulo.tex E risc-v.tex %%
%%%%%%%%%%%%%%%%%%%%%%%%%%%%%%%%%%%%%%%%%%%%%%%%%%%%%%%%%%%%%%%%%
\usepackage[T1]{fontenc} % hifenização (quebra de palavra)
\usepackage[utf8]{inputenc} % caracteres acentuados
\usepackage[brazil]{babel} % tradução inglês -> português; e. g., resumo ao invés de abstract

\usepackage{graphicx}	% possibilita inserção de imagem/pdf no texto
\usepackage{rotating}	% permite a inserção de figuras deitadas
\usepackage{float}		% dá a opção [H] para inserção precisa de figura
\usepackage{booktabs}	% fornece comandos úteis para a confecção de tabelas
\usepackage{multirow}	% para mesclagem de células verticais numa tabela

\usepackage{amsmath, amssymb}	% suporte para equações matemáticas 

\usepackage{mathptmx}		% fonte times no corpo de texto
\usepackage[scaled]{helvet}	% helvetica em seções, subseções, legendas, etc.

\usepackage{titlesec} % pacote para modificações nas strings de seções e similares
\titleformat*{\section}			{\bfseries\sffamily\scshape}	% altera a fonte e estilo do nome de seção
\titleformat*{\subsection}		{\bfseries\sffamily\scshape}	% altera a fonte e estilo do nome de subseção
\titleformat*{\subsubsection}	{\sffamily\scshape}				% altera a fonte do nome de subsubseção
% 0 pt -> identação esquerda, *2 -> espaço acima do nome, *.5 -> espaço abaixo
\titlespacing*{\section}		{0pt}{*2}{*.5}	% modifica o espaçamento do nome de seção
\titlespacing*{\subsection}		{0pt}{*2}{*.5}	% modifica o espaçamento do nome de subseção
\titlespacing*{\subsubsection}	{0pt}{*2}{*.5}	% modifica o espaçamento do nome de subsubseção

% definição de vários comprimentos importantes da página e texto
%% comprimentos em polegadas (in) para ter numeros redondos
\setlength\paperheight 		{11in}		% altura da folha
\setlength\paperwidth  		{8.5in} 	% largura da folha
\setlength{\textheight}		{9.25in}	% altura do texto
\setlength{\topmargin}		{-0.7in}    % reajuste de margens
\setlength{\headheight}		{0.20in}
\setlength{\headsep}		{0.25in}
\setlength{\footskip}		{0.50in}
\flushbottom
\setlength{\textwidth}		{7in}		% largura do texto
\setlength{\oddsidemargin}	{-0.25in}
\setlength{\evensidemargin}	{-0.25in}
\setlength{\columnsep}		{2pc}		% separação entre colunas (1 in ~ 6 pc)

\usepackage{caption} 	% permite mudar configurações de legendas
\captionsetup{
	labelfont={sf, bf},	% legenda em helvetica e negrito
	textfont=it,		% texto da legenda em itálico
	format=hang			% linhas identadas (observável apenas em textos longos)
}

\usepackage{lastpage}	% cria macro para o número da última página
\usepackage{fancyhdr}	% fornece comandos que facilitam enfeitar a página
	%% informações no topo 
	\fancyhead[L]{Turma 3} 							% à esquerda
	\fancyhead[C]{2022/2}							% ao centro
	\fancyhead[R]{Prof. Dr. Marcus Vinicius Lamar}	% à direita
	%% informações no fundo
	\fancyfoot[L]{CIC0099 -- Organização e Arquitetura de Computadores} % à esquerda
	\fancyfoot[C]{}														% ao centro (nenhuma)
	\fancyfoot[R]{\thepage/\pageref*{LastPage}}							% à direita
	                       %^ número da última página 
	%% linhas horizontais superior e inferior
	             % comprimento   % altura
	\renewcommand{\headrulewidth}{0.5pt}	% mude para 0 pt se quiser apagar a linha
	\renewcommand{\footrulewidth}{0.5pt}	% mude para 0 pt se quiser apagar a linha 
	
	\pagestyle{fancy} % doravante, todas a páginas, exceto o título, terão os enfeites definidos acima

\usepackage{xcolor}		% maior variedade de cores
\usepackage{listings} 	% inserção de texto de código em Latex
	% definição de cores
\definecolor{azulunb}{cmyk}{1, 0.65, 0, 0.35}
\definecolor{verdeunb}{cmyk}{ 1, 0,    1, 0.2}
% language definition
\lstdefinelanguage[RISC-V]{Assembler}
{
	alsoletter={.}, % allow dots in keywords
	alsodigit={0x}, % hex numbers are numbers too!
	morekeywords=[1]{ % instructions and pseudoinstructions 
		csrr, call,
		divu, mul, mv, remu,
		lb, lh, lw, lbu, lhu, li, la,
		sb, sh, sw,
		sll, slli, srl, srli, sra, srai,
		add, addi, sub, lui, auipc,
		xor, xori, or, ori, and, andi,
		slt, slti, sltu, sltiu,
		beq, beqz, bne, blt, bge, bltu, bgeu, bgt,
		j, jr, jal, jalr, ret,
		ecall, ebreak, nop
	},
	morekeywords=[2]{ % sections of our code and other directives
		.include, .eqv,
		.align, .ascii, .asciiz, .string, .byte, .data, .double, .extern,
		.float, .globl, .half, .space, .text, .word
	},
	morekeywords=[3]{ % registers
		zero, ra, sp, gp, tp, s0, fp,
		t0, t1, t2, t3, t4, t5, t6,
		s1, s2, s3, s4, s5, s6, s7, s8, s9, s10, s11,
		a0, a1, a2, a3, a4, a5, a6, a7,
		ft0, ft1, ft2, ft3, ft4, ft5, ft6, ft7,
		fs0, fs1, fs2, fs3, fs4, fs5, fs6, fs7, fs8, fs9, fs10, fs11,
		fa0, fa1, fa2, fa3, fa4, fa5, fa6, fa7
	},
	morecomment=[l]{;},   % mark ; as line comment start
	morecomment=[l]{\#},  % as well as # (even though it is unconventional)
	morestring=[b]",      % mark " as string start/end
	morestring=[b]'       % also mark ' as string start/end
}

% usage example:

% define some basic colors
\definecolor{mauve}{rgb}{0.58, 0, 0.82}
\makeatletter
\DeclareRobustCommand\em
{\@nomath\em \ifdim \fontdimen\@ne\font >\z@
	\eminnershape \else \slshape \fi}%
\makeatother
\lstset{
	% listings sonderzeichen (for german weirdness)
	literate={ö}{{\"o}}1
	{ä}{{\"a}}1
	{ü}{{\"u}}1,
	backgroundcolor=\color{gray!10},
	commentstyle=\em\color{verdeunb},  % comments are green
	basicstyle=\small\ttfamily,                    % very small code
	breaklines=true,                              % break long lines
	keywordstyle=[1]\color{azulunb},        % instructions are blue
	keywordstyle=[2]\color{orange!80!black},      % sections/other directives are orange
	keywordstyle=[3]\color{red!75!black},         % registers are red
	stringstyle=\color{mauve},                    % strings are from the telekom
	identifierstyle=\color{teal},                 % user declared addresses are teal
	frame=lr,                                      % black line on the left side of code
	language=[RISC-V]{Assembler},                   % all code is RISC-V
	tabsize=2,                                    % indent tabs with 4 spaces
	showstringspaces=false                        % do not replace spaces with weird underlines
}		% suporte para o Assembly RISC-V

% comando \teaser para a criação da figura opcional no título
\newcommand{\teaser}[2]{ % recebe dois argumentos, uma figura e uma legenda
	\begin{center}
		\captionsetup{type=figure}
		\includegraphics[width=5cm]{#1} % figura é o primeiro argumento,
		\captionof{figure}{#2}			% legenda é o segundo
	\end{center}
}

\usepackage[colorlinks]{hyperref} % este pacote gera a funcionalidade de hyperlinks, citações, etc. e para evitar erros deve ser carregado por último % arquivo .tex que define vários parâmetros

% título do pdf
\title{\sffamily\bfseries
	Modelo sugerido para os relatórios\\
	e obrigatório para a documentação do projeto final
}
% autores
\author{%
	João Pedro G.C.M. Antonow	\thanks{221006351@aluno.unb.br}	\and
	Caleb Martim de Oliveira	\thanks{221017060@aluno.unb.br} 	\and
	Hiago Sousa Rocha		\thanks{221002049@aluno.unb.br} 		\and
	Luca Heringer Megiorin	\thanks{victor.lisboa@aluno.unb.br}
 \and
	Luis Augusto da Silveira Cavalcanti\thanks{victor.lisboa@aluno.unb.br}
}
% afiliação e imagem de amostra
\date{Universidade de Brasília,15 de abril de 2024\\
	\teaser{figuras/paulista1891}{Imagem de teaser (opcional)}
}

\begin{document}
	\maketitle	% imprime o título
	\abstract{% abstract/resumo do coumento
		Faça aqui uma breve descrição do que foi feito em cada seção e como elas se relacionam.
		Uma seção pode ser referenciada automaticamente usando o comando \verb|\ref{sec:label da seção}|, como mostra o texto abaixo. 
		Esse comando é usado de maneira geral para referenciar a equações, figuras, tabelas e códigos.\\
		
		\textit{``Uma dificuldade desnecessária dos alunos de OAC é o template de relatório/documentação obscuro ou altamente técnico. É o objetivo deste novo template criar um modelo mais legível e prático, e para tanto foram dedicadas três seções;
		na Seção \ref{sec:formulas}, explica-se como usar o básico das fórmulas matemáticas em \LaTeX~e como inserir figuras, tabelas e código RISC-V;
		na Seção \ref{sec:hyperlinks}, apresenta-se rapidamente como criar URLs, hyperlinks e citações a referências clicáveis;
		finalmente, na Seção~\ref{sec:casos}, são oferecidas algumas sugestões de como incluir tabelas ou figuras grandes.''}
		
		\smallskip
		\noindent
		\textbf{\sffamily Palavras-chave:} 
		OAC $\cdot$
		Assembly IRSC-V $\cdot$
		Template
	}
	
	\section{Fórmulas matemáticas}
	\label{sec:formulas}
		Uma nova seção pode ser feita simplesmente digitando \verb|\section{Nome da seção}|. 
		Esta simplicidade reaparece nas subseções e subsubseções, como veremos.
		Entretanto, antes disso, apresentemos como colocar fórmulas matemáticas; basta usar o ambiente {\tt equation}
		%
		\begin{equation}
			\label{eq:euler}
			\sum_{n=1}^\infty \frac{1}{n^2} = \frac{\pi^2}{6}
		\end{equation}
		%
		para escrever uma equação enumerada, ou o ambiente {\tt align} para múltiplas equações numeradas:
		%
		\begin{align}
			\tan(1/x) &= \frac{1}{\tan(x)} 				\label{tan1/x}	\\
			        &= \cot(x) 											\\
			        &= \tan\left(\frac{\pi}{2}-x\right) \label{pi/2-x}	\\ \implies
			      1/x &= \frac{\pi}{2}-x + 2k\pi, 		\quad k\in\mathbb{Z}.
		\end{align}	
		% 
		Repare no uso do {\tt\&} e \verb|\\| dentro do código para alinhar as equações. 
		Elas também podem ser referenciadas pelo comando \verb|\ref{}| assim como as seções, desde que você tenha criado uma label antes. Exemplo:
		
		\textit{``A equação \ref{eq:euler} é uma famosa descoberta de Leonhard Euler.''}
			
		Particularmente, equações também podem ser referenciadas por \verb|\eqref{}|:
		
		\textit{``A equação \eqref{eq:euler} é uma famosa descoberta de Leonhard Euler.''}	
		
		Para o caso do \verb|align|, é necessário criar uma \verb|\label{}| para cada equação citada.
		
		\textit{``\eqref{tan1/x} parece um problema difícil até chegarmos em \eqref{pi/2-x}.''}	
			
		Caso a numeração das equações não seja desejada, coloque {\tt equation*} e {\tt ailgn*} no lugar de {\tt equation} e {\tt align}, respectivamente.
		Finalmente, se não for desejado iniciar a fórmula em nova linha, use \verb|$$| ou \verb|\(\)| no lugar dos ambientes.\\
		Por exemplo,
		$x+y=2$ e \(\sin^2\theta+\cos^2\theta=1 \forall \theta\in\mathbb{R}\).	
		%		
			
		\subsection{Inserção de figura, tabela e código}
			Uma nova seção pode ser feita simplesmente digitando \verb|\subsection{Nome da seção}|. 
			
			Anteriormente, mostrou-se o básico do uso de matemática no \LaTeX, mas em OAC o interesse maior está em exibir figuras, tabelas, e, eventualmente, códigos RISC-V.
			
			\subsubsection{Figura}
				Colocar uma figura é bem simples:
				use o comando 
				
				\verb|\includegraphics[width=comprimento]{dir/nome}|
				
				dentro do ambiente {\tt figure}. Por exemplo,
				%
				\begin{figure}[H]\centering
					\includegraphics[width=5cm]{figuras/logo_unb}
					\caption{legenda}
				\end{figure}
				%
				O {\tt [H]} e \verb|\centering| no código servem para fixar a posição da figura e centralizá-la, nessa ordem.
				Repare também na facilidade em escrever uma legenda através do comando \verb|\caption{}|.
				Os tipos de arquivos de imagem aceitos vão de {\tt.png}'s a {\tt.pdf}'s.
				%
			%
			
			\subsubsection{Tabela}
				O código de criação da tabela se torna tão complicada quando mais complexa for o design desejado da tabela. 
				A tabela abaixo é simples
				%
				\begin{table}[H]\centering
					\begin{tabular}{c|cccc}
						\toprule
						ISA & ALMs & Regs & MEM & DSPs \\
						\midrule
						\midrule
						RV32I   &  3271 & 1616 &     0 & 0  \\
						RV32IM  &  8070 & 1616 &     0 & 12 \\
						RV32IMF & 11341 & 4126 & 47616 & 18 \\
						\bottomrule
					\end{tabular}
					\caption{%
						Exemplo de tabela simples. 
						Repare nos símbolos dentro do código usados para alinhar as células.
					}
					\label{tab:simples}
				\end{table}
				%
				enquanto a tabela abaixo já é mais complicada.
				%
				\begin{table}[H]\centering
					\def\arraystretch{1.2} % expande a tabela
					\begin{tabular}{c|cc}
						\toprule
						\multirow{2}{*}{$n$} & 
						\multicolumn{2}{c}{$I_1$} \\
						\cline{2-3}
						& teórico & real \\ 
						\hline\hline
						40 & 14256 & 14256 \\
						50 & 22316 & 22316 \\
						60 & 32176 & 32176 \\
						70 & 43836 & 43836 \\
						80 & 57296 & 57296 \\
						90 & 72556 & 72556 \\
						100 & 89616 & 89616 \\
						\bottomrule
					\end{tabular}
					\caption{Exemplo de tabela mais complexa.}
					\label{tab:complexa}
				\end{table}
				%
				Tanto a tabela~\ref{tab:simples} quanto a~\ref{tab:complexa} foram criadas usando o ambiente {\tt tabular} dentro do ambiente {\tt table}, e em seguida definindo através de \verb|c|'s quantas colunas existem na tabela e quais delas devem ter uma linha vertical.
				Ao final, os comandos de mescla de células 
				\verb|\multirow| e \verb|\multicolumn|
				foram usados para melhorar a estética do todo.
				%
			%
			
			\subsubsection{Código RISC-V}
				Embora o grosso dos códigos relevantes aos relatórios e trabalho seja entregues em {\tt.s} ou {\tt.asm} à parte, as vezes é necessário fazer um comentário sobre um pequeno trecho, e nessa ocasião é cômodo ter o devido suporte no \LaTeX.
				
				Podemos inserir um código Assembly RISC-V no texto de duas formas:
				(i) escrevendo o código linha a linha ou 
				(ii) importanto o arquivo com o código.
				Para o primeiro caso, usa-se o ambiente {\tt lstlisting}.
				%
				\begin{lstlisting}[caption={Trecho de código Assembly RISC-V}]
				.text
				
					GameLoop:
					la	  a0, personagem
					call	Animacao
					j	    GameLoop					
					
					.include "Animacao.s"
				\end{lstlisting}
				%
				Para o segundo, usa-se o comando \verb|\lstinputlisting{dir/nome.s}|.
				%
				\lstinputlisting[caption={Código importado da pasta listings},
				                 label=listing]
				                {listings/Sleep.s}
				%
				Assim como o resto das referências até agora, você pode referenciar um listing dando \verb|\ref{}| na label apropriada. 
				
				\textit{``O listing~\ref{listing} é uma função de sleep."}
				%
			%
		%
	%
	
	\section{Hyperlinks, urls e citações}  
	\label{sec:hyperlinks}
		Até aqui, criamos hyperlinks apenas para objetos internos ao arquivo pdf.
		Em geral, podemos criar hyperlinks para páginas na web escondidos em palavras/strings ou importar URLs inteiras, assim como fazer citações do campo de bibliografia.
		
		Para introduzir uma URL, basta usar o comando \verb|\url{}|. 
		Por exempolo,
		
		\url{https://www.youtube.com}
		
		o levará para o Youtube.
		Alternativamente, você também pode utilizar o comando \verb|\href{url}{string}| para esconder o texto da URL numa palavra ou frase. 
		
		\textit{``Clique \href{https://aprender3.unb.br/login/index.php}{aqui} para entrar no aprender3."}
	          
		Por último, a listagem da sua bibliografia é convenientemente citada através do comando \verb|\cite{}|.		
				
		\textit{``Já foi demonstrado~\cite{taylor} que o erro...''}	
		
		\textit{``\cite{tipler} é um excelente livro de Física.''}						
				
	% Bibliografia -- listagem de referências		
	\begin{thebibliography}{2} % número de itens
		\bibitem{tipler} 															% nome para citação
			Mosca, Gene e Tipler, Paul A. 											% autores
			\textit{Física Volume 2, 5\textordfeminine Edição}. 					% obra/referência
			LTC--Livros Técnicos e Científicos Editora S.A., Rio de Janeiro, 2006. 	% editora, data, etc.
		
		\bibitem{taylor}												% nome para citação
			Taylor, John R.												% autores
			\textit{An Introduction to Error Analysis, Second Edition}.	% obra/referência
			University Science Books, Sausalito (CA), 1997. 			% editora, data, etc. 
	\end{thebibliography}
				
	\section{Casos especiais}	
	\label{sec:casos}
		Nem sempre é possível inserir a imagem no tamanho desejado, especialmente quando o documento se encontra, como agora, em forma de coluna dupla. 
		É claro, a mesma situação se estende para as tabelas.
		Nesses casos, considere colocar o arquivo em formato de coluna única através do comando 
		\verb|\onecolumn|, para aí colocar a(s) figura(s)/tabela(s) desejadas.
		Feito isso, retorne ao modo coluna dupla com \verb|\twocolumn|.\\
		
		\textbf{Obs.:} ambos os comandos iniciam uma página em branco.\\
		
		Se uma imagem ou tabela ocupa toda uma página, considere utilizar o ambiente {\tt sidewaysfigure} ou {\tt sidewaystable} após \verb|\onecolumn|, respectivamente.
		
		\onecolumn
		Imagens longas como a de baixo são melhor posicionadas em coluna única.
		
		\begin{figure}[H]\centering
			\includegraphics[width=\textwidth]
			{figuras/forma de onda.png}
			\caption{%
				Figura comprida
			}
		\end{figure}   
		
		Tabelas grandes também se beneficiam da coluna única.
		
		\begin{table}[H]\centering
			\def\arraystretch{1.2}
			\begin{tabular}{|c|c|c|}
				\hline
				Operação & Funcionalidade & Código\\
				\hline\hline
				OPAND & Faz a operação lógica AND entre dois números de 32 bits & 00000\\
				\hline
				OPOR & Faz a operação lógica OR entre dois números de 32 bits & 00001\\
				\hline
				OPXOR & Faz a operação lógica XOR entre dois números de 32 bits & 00010\\
				\hline
				OPADD & Faz a operação de soma entre dois números de 32 bits & 00011\\
				\hline
				OPSUB & Faz a operação de subtração entre dois números de 32 bits & 00100\\
				\hline
				& Põe o resultado com o valor lógico da expressão $iA < iB$, &\\
				OPSLT & isto é, se a expressão for verdadeira o resultado é um & 00101 \\
				&se for falsa o resultado é zero &\\
				\hline
				& seta o resultado com o valor logico da expressão $iA < iB$ , isto é, se a expresão & \\
				OPSLTU & for verdadeira o resultado é um se for falsa o resultado é zero, & 00110\\
				& mas desconsidera o sinal dos números iA e iB &\\
				\hline
				\multirow{2}{*}{OPSLL} & Desloca os bits de iA para a esquerda em até 31 posições & 
				\multirow{2}{*}{00111}\\
				& colocando zeros nos bits novos resultantes do deslocamento &\\
				\hline
				\multirow{2}{*}{OPSRL} & Desloca os bits de iA para a direita em até 31 posições & 
				\multirow{2}{*}{01000}\\
				& colocando zeros nos bits novos resultantes do deslocamento &\\
				\hline
				\multirow{2}{*}{OPSRA}  & desloca os bits de iA para a direita em até 31 posições & \multirow{2}{*}{01001}\\
				& conservando o sinal do número que foi deslocado &\\
				\hline
				OPLUI & carrega como resultado os 32 bits de iB & 01010\\
				\hline
				\multirow{2}{*}{OPMUL} & Faz a operação de multiplicação entre dois números de 32bits & 
				\multirow{2}{*}{01011}\\
				& e pega os 32 primeiros bits resultantes da multiplicação [31:0] & \\
				\hline
				\multirow{2}{*}{OPMULH} & Faz a operação de multiplicação entre dois números de 32bits & 
				\multirow{2}{*}{01100}
				\\
				& e pega os 32 últimos bits resultantes da multiplicação [63:32] &\\
				\hline
				\multirow{2}{*}{OPMULHU} & Faz a operação de multiplicação entre dois números de 32bits & 
				\multirow{2}{*}{01101}\\
				& sem sinal e pega os 32 últimos bits resultantes da multiplicação [63:32] & \\
				\hline
				& Faz a operação de multiplicação entre dois números de 32bits &\\
				OPMULHSU & um deles com sinal e o outro sem sinal e pega os & 01110\\
				& 32 últimos bits resultantes da multiplicação [63:32] &\\
				\hline
				OPDIV & Faz a operação de divisão entre dois números de 32 bits & 01111\\
				\hline
				OPDIVU & Faz a operação de divisão entre dois números de 32 bits sem sinal & 10000\\
				\hline
				OPREM & Faz a operação de resto da divisão entre dois números de 32 bits & 10001\\
				\hline
				OPREMU & Faz a operação de resto da divisão entre dois números de 32 bits sem sinal & 10010\\
				\hline
				OPNULL & Retorna como resultado o número zero com 32 bits & 11111\\
				\hline
			\end{tabular}
			\caption{Tabela grande.}
		\end{table}
		
		Em sequência, será colocada na próxima página uma figura deitada. 
		Para iniciar nova página em branco, escreva \verb|\newpage| ou \verb|\clearpage|.
		
		\newpage
		\begin{sidewaysfigure}\centering
			\includegraphics[width=\paperwidth]
			{figuras/máquina de estados.pdf}
			\caption{%
				Figura grande. Note a mudança de {\tt width=Xcm} para {\tt paperwidth}.
			}
		\end{sidewaysfigure}
		
		
\end{document}